\section{Study of Cognition, Adolescents and Mobile Phones (SCAMP)}

- start in 2014
- 7000 young people
- 39 schools in Greater London
- ~2 years between followups
- brain function, digital tech use, demographics, socio-economic status, lifestyle, environment, physical and mental health

SCAMP Study results so far
- who takes part and who doesnt
- teens, social media and wellbeing
- screentime before bed and quality of sleep
- kids usage patterns of mobile phones

\section{Innovating UK clean air policies to prevent cognitive disorders (InSPIRE)}

\textbf{Programme A:}
WP1-2 Putting Policy into Prevention
WP3 Build and share high impact policy evaluation toolkit and scenario simulation dashboards

\textbf{Programme B:}
WP4 Linking air quality with social determinants and health inequalities
WP5 Building Historical Model of Air Pollution
WP6 Linking air quality with cognitive incomes

We will pioneer the first UK model of PM2.5 for 1970-2020 (WP5); link it to cognitive outcomes (e.g., dementia, neurodegenerative disease, early-life cognitive development, mental disorders) for the 1946, 1958, 1970, millennium cohorts (WP6) based on a complex systems model of how upstream social determinants (e.g., social disadvantage, risk exposure) and health inequalities (e.g., life expectancy, healthcare access) influence PM2.5‘s impact on cognitive health (WP4).

Research Questions:
Modifiable social determinants and health inequalities (WP4): From a complex systems perspective, what social determinants and health inequalities are associated with the worst-to-best 50-year PM2.5 trends and cognitive health outcomes? How do impacts differ based on conurbation and local authority level differences in social determinants, inequalities and the complex social environments in which people live? How do these impacts further differ by gender, ethnicity and socioeconomic status?

Air pollution modelling (WP5):
What are the quantitative exposure estimates for all UK local authorities at key historical timepoints between 1970 and 2020? And can we extend this model into 2050?
What are the different air quality trends for PM2.5 across the UK between 1970-2020?

Linking PM2.5 to cognitive health and related brain and respiratory disease (WP6): Which cognitive outcomes (e.g., dementia, ADHD, depression, etc) and neuro-developmental milestones (e.g., language, psychomotor, spatial memory) are impacted by PM2.5 and at what concentrations and durations of exposure across the life course? What are the mechanistic pathways by which PM2.5 exposure impacts cognition? How do insights into these pathways allow us to explore related air pollution diseases?

\textbf{Programme C:}
WP7-9 Evaluating primaty prevention health policy

Publications:
see mendeley