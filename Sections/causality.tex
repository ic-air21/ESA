\chapter{Causality Analysis}
\section{Abstract}

\section{Background and Literature Review}
\cite[p.550]{Gelfand2019HandbookStatistics}

%An extension of exposure-response models is to casual models with Causal Expsoure-Repsonse Functions (CERFs), where an analysis of the evidence of causality can be used to more accurately represent the real-world effects of air pollution exposure on mental health outcomes. Assessing the evidence of causality in epidemiology can be approached in numerous ways, a traditional criteria for evidence of causality comes from Sir Austin Bradford Hill in 1956 \citep{Hill1965TheCausation}: Strength, Consistency, Specificity, Temporality, Biological Gradient, Plausibility, Coherence, Experiment, and Analogy. More modern approaches have since been devised, with a similar approach involving Directed Acyclic Graphs (DAGs) set out in \cite{Shimonovich2021AssessingThinking} or Empirical Dynamical Models (EDMs), often used in environmental modelling, as in \cite{Wu2020EmpiricalChina}. I will assess each indicator of causality using a variety of techniques: using statistical measures, biological context, and existing evidence. For example, reviewing the current evidence, I hope to identify any plausible biological pathways that could show how air pollution can adversely effect mental health outcomes for some indication of causality, such as inflammatory pathways discussed in \cite{Bakolis2021MentalSurvey}. As for strength, temporality and biological gradient, these can be assessed using common statistical methods and by assessing the exposure-response function. The investigation of spatial causal effects is a particular novel area of causal research and extending the notions of assessing causality to spatiotemporal data may open up new or extended areas of casual inference \citep{Akbari2021SpatialInference}. Then using any evidence that is found for causation, appropriate CERFs can be defined in a similar way to the original ERFs, examples of these can be seen in \cite{Ren2021BayesianOutcomes} and \cite{Papadogeorgou2020AMatter}, where development and assessment of the CERFs is analogous to the ERF case. In a similar vein to the exposure-response model, the proposed new CERFs and modelling framework will incorporate many of the cutting-edge approaches for multi-exposures, space-time effects and interactions, and the surrounding contextual covariates. 

%Another important consideration in these models is the causal evidence behind the effect modifiers and potential confounders. It will be interesting to explore the causation arguments for the proximity to greenspaces and the previous evidence for the effects of underlying health, genetics, and socioeconomic status on mental health. This analysis and inference should give insight into the realistic effects of these covariates and how they tie into the relationship between air pollution and mental health. Finally, by considering the causal evidence and effect, the model can more accurately represent the real world situation in context to be used in prediction and policy assessment. From this, direct causal effects of air pollution can be measured, and as part of this project,  trends in vunerable groups and areas can be identified. Some papers that have investigated the proposed connection between causal inference and public health policy include \cite{Glass2013CausalHealth} and \cite{Pan2016HealthIntervention}.


\section{Research Methods}

\section{Expected Results and Significance}