\chapter{Registration}

This project will develop an interdisciplinary, quantitative approach to quantify the effects of air pollution on mental health. Using mathematical methodologies for data analysis, the project will include a Bayesian modelling framework, and Machine Learning classification techniques for land use. All while taking into consideration the context of the data and results, supported by background information in biological and psychological pathways for mental health disorders, and I will start to look at some implications into policy and governmental strategies.

\subsection{Data}
The National Oceanic and Atmospheric Administration (NOAA) Integrated Surface
Database (ISD) \citep{NationalCentresforEnvironmentalInformationGlobalISD} provides accurate, up-to-date, and high spatial and temporal resolution measurements of temperature and weather, which can be downloaded using the worldmet R package. The Met Office also provides air quality numerical models which aim to represent the underlying physical processes of air pollution emissions, \citep{AirOffice}.

For observed ground air pollution, the UK National Monitoring Network provides measurements for a range of different air pollutants \citep{DepartmentforEnvironmentMonitoringUK}. For this project I have chosen 4 main pollutants: Ozone (O\textsubscript{3}), fine particulate matter (PM\textsubscript{2.5}), Nitrogen Dioxide (NO\textsubscript{2}), and Sulphur Dioxide (SO\textsubscript{2}). I have chosen these to represent the main sources of pollution to encapsulate a thorough picture of air pollutants in the UK. SO\textsubscript{2} and NO\textsubscript{2} are mainly industry-related pollutants, closely related to Nitrogen Oxides, and have been previously linked to a number of different negative health effects \citep{Azimi2018Air2015}. Whereas, O\textsubscript{3} is the product of reactions between UV light and primarily emissions from the burning of fossil fuels such as Nitrogen Dioxide, Nitrogen Oxide, and Volatile Organic Compounds, this introduces an interesting relationship between meteorological conditions, climate change and air pollution. Finally, PM\textsubscript{2.5} is a complicated mixture of organic and inorganic particles from a variety of different sources and due to the choice of fine particle size it is easily breathed into the lungs and into the bloodstream \citep{Bell2007SpatialStudies}. PM\textsubscript{2.5} also has an existing research base in cognitive function and decline \citep[e.g.][]{Delgado-Saborit2021APopulation}. The R package openair will be useful for exploratory analysis  of air pollution data.

As for the satellite imagery data, for temperature and other climate measures such as wind direction, humidity and terrain, the ERA5 datasets provide hourly reanalysed estimates on a 9km grid or 30km grid \citep{Copernicus2019ERA5-LandPresent}. For air pollution, the aforementioned NASA generated pollution estimates can be used, with direct measurements for the required Ozone (O\textsubscript{3}), Nitrogen Dioxide (NO\textsubscript{2}), and Sulphur Dioxide (SO\textsubscript{2}), and indirect measures for fine particulate matter through reflectance, aerosol index and aerosol optical depth. Each of these measurements have different spatial resolution, generally around 375m-2km, and a daily or twice daily temporal resolution. Then for the land use imagery I will use the visible light and near infrared measurements from NASA's open source data from the Landsat satellites, at a 30m spatial resolution and 8-day temporal resolution. 


Finally, the data for the mental health outcomes with location, lifestyle, genetic data and socioeconomic context will be from the aforementioned UK Biobank cohort dataset, which includes 500,000 adults aged 40-69 at recruitment, incorporating demographic, lifestyle, personal and family medical history, genetic data and imaging. This project will be a novel use of this dataset, utilising the new hybrid pollution estimates along with being the first Bayesian use of the data in exposure modelling, public health surveillance, and mental health research. Another consideration for further investigation is the Millennium Cohort Study, that follows around 19,000 young people born in 2000-2002. The use of this cohort study will be useful in investigating whole life course exposures to air pollution and explore its relationship to mental health within children and young people.

%The available variables, exact amount and spread of participants, and the temporal and spatial resolution of the data will determine the exposure and epidemiological model inputs and the resolution of all the models required, so this will form part of the model considerations.

\subsection{Modelling and Analysis}

\subsubsection{Objective 1a}
Objective 1 involves comparing methods of data assimilation for air pollution measurements. Within a statistical framework there are two main approaches, as identified in \citet[p.133-151]{Gelfand2019HandbookStatistics}: joint modelling and regression-based. A popular example of a joint modelling method is Bayesian Melding as proposed in \cite{Fuentes2005ModelModels} and a common regression-based approach is downscaling as in \cite{Berrocal2010AModels}. I will also be particularly interested in the extensions of these methods, such as in \cite{Choi2009Spatial-temporalMortality} for spatiotemporal air pollution, \cite{Berrocal2010AMisalignment} for multiple air pollutants, and methods such as \cite{Hoogh2019PredictingSwitzerland} and \cite{Harnisch2016ErrorSystem} for missing data due to cloud cover on satellite imagery. Ensemble Kalman filtering is also one technique of particular interest, with the direct extension to non-linear spatiotemporal processes \citep{Roth2017ThePerspective}. I  will then assess the models' performance in a similar way to \cite{Denby2008ComparisonScale}, using comparisons of cross-validation statistics.


%The approach to incorporating the underlying numerical physical model is also a consideration, the Met Office does supply these models for the UK but it would be interesting to explore the methodologies behind these.

%Model analysis would then be used to determine the relationships between the air pollutants and meteorological measurements, such as temperature, humidity and wind. 

Meteorological measurements, such as temperature, humidity and wind, have a known effect on air pollutant concentrations. So identifying and quantifying some of the more notable patterns will help determine any effect modification or confounding they may have in the proposed models of Objectives 3 and 4.

\subsubsection{Objective 1b}
For the air pollution exposure model we need to consider an individual's exposure to each of the air pollutants:  PM\textsubscript{2.5}, O\textsubscript{3}, NO\textsubscript{2} and SO\textsubscript{2}. The exposure of a participant is taken to be the cumulative long-term amount of the air pollutants they are in contact with over a certain time period. The exposure effects may also have a temporal lag to them, it may be months, or years before changes in air pollution concentrations have a noticeable effect on mental health measures. So, it would be interesting to include a lag structure within the exposure model for later choices of lag in the epidemiological model, in a similar way to \cite{Richardson2011HierarchicalAssociations}.

Another consideration will be given to the spatial misalignment between the resolution of the exposure model and the resolution of the health outcomes. Home locations are given at a resolution of 100m or 1km in the Biobank dataset, so I will use the continuous air pollutant estimates modelled in Objective 1a to give exposure estimates at a matched resolution. High spatial resolution will be beneficial for later analysis, prediction, and potential policy considerations.

Finally, assessment of the variance and errors of the air pollution exposure estimates will be insightful. \cite{Wu2019MethodsExposures} discusses the main sources of errors in exposure assessment and four statistical approaches to account for these. In a similar way, I will evaluate the potential sources for errors in my model and investigate the Bayesian approaches to accounting for these uncertainties in the later epidemiological model.

\subsubsection{Objective 2} 
For the estimation of proximity to greenspaces I will use machine learning classification algorithms to identify different land use areas in the UK, particularly the accurate identification of greenspaces is required. There are three commonly used classification methods: unsupervised k-means, supervised CART and supervised Random Forest, and I will assess their accuracy and precision for the UK landscape. R packages RStoolbox, landsat and hsdar are particularly useful to process, analyse and transform the satellite imagery into usable data for these algorithms and further uses. From the classified land use, measurements of greenspace can be derived. The Normalized Difference Vegetation Index (NDVI) is a common estimate for the density of green in a certain area which I aim to then use to assess the extent of green within certain radii of a residence, this will be done in a similar way to \cite{Maes2021BenefitHealth} and \cite{Su2019AssociationsResolutions}, with a range of buffer areas: 50m, 100m, 250m, 500m. 

\subsubsection{Objective 3} 
Before incorporating the health data into the model, it is important to assess it separately. First, an assessment of the participant sample will help to give a contextual basis of our analyses within the UK population. I am interested in the makeup of the participants age, sex, location, socioeconomic status, occupation, ethnicity, lifestyle, health status etc., particularly for vulnerable or underrepresented groups. I am also interested in identifying any clear trends in the mental health outcome data from the offset, maybe spatially, temporally or demographically, that can then potentially be explained via the air pollution or meterological variables. As for the definition of mental health outcomes/disorders, corresponding to the available UK Biobank variables, I will use the self-reported mental health indicators for depression, anxiety, and bipolar, along with the physician-derived depression, anxiety, and bipolar statuses. These sub-clinical measures will be interesting to assess the underlying mental health of the population, compared to acute mental health events, such as given by the diagnosed disorders in the Biobank data using the International Statistical Classification of Diseases and Related Health Problems (ICD) codes \citep{WorldHealthOrganisationICD-10Version:2019} for inpatient hospital visits.

Then for the exposure-response modelling, I will review and compare exposure-response estimation methods within a hierarchical Bayesian framework to then investigate a novel approach utilising the cutting-edge methods in multivariate spatiotemporal modelling with extensive contextual variables. I will start off by reviewing single-exposure effects, looking at each of the four pollutants separately and assessing the one-dimensional approaches first. A common basic approach is Generalised Linear Models (GLMs) with a linear Exposure-Response Function (ERF), such as the one seen in \cite{Shi2016Low-ConcentrationStudy}. The GLM approach can then be extended to non-linear Exposure-Response Functions (ERFs), for example, by using splines: \cite{Smith2000ThresholdArizona} gives an example of the use of B-splines and \cite{Daniels2000EstimatingCities} for cubic splines, both using a cross-validation method for the estimation of the splines. Generalised Additive Models (GAMs) are then a natural progression to more complicated relationship modelling, with similar linear and non-linear approaches available, again this can be seen in \cite{Smith2000ThresholdArizona}. Furthermore, I will investigate the lag effects of the exposure on the response, in a similar way to \cite{Shi2016Low-ConcentrationStudy}, with considerations on how we chose a lag through Goodness-of-fit tests as in \cite{Richardson2011LaggingAnalyses}. Finally, the extension to simultaneous exposure health effects should be considered, there are again a number of different approaches to this, but within the Bayesian framework I will mainly be interested in ERFs in regression models. An extension to multi-pollutant spline functions can be seen in \cite{Chen2013InfluenceChina} and a similar kernel function can be seen in \cite{Bobb2015BayesianMixtures}. The potential effect modifying effect of proximity to greenspaces can be easily incorporated into these regression models, where the effect of its inclusion can be assessed in a similar way to as in \cite{Ghosh2010PaternalMothers} and \cite{Mariet2021AssociationExposure}. Similarly, the previous considerations for the confounding and effect modifying properties of meteorological measurements (including temperature, humidity and wind) can be investigated through the model. For the purposes of this project, the air pollution and meteorological measurements are being considered as a proxy to direct measures of climate change: CO\textsubscript{2} and CO emissions are closely linked to NO\textsubscript{2} and SO\textsubscript{2} emissions \citep{Chen2007OutdoorEffects}; long-term temperature (or climatic temperature) increase is a common indication of global warming \citep[e.g.][]{Du2019ChangesHiatus}; and the levels of atmospheric Ozone are intricately linked to the greenhouse effect \citep{Meleux2007IncreaseChange}.

The novelty of my approach will come from the combination of the existing approaches from all directions and the extension to the inclusion of underlying health conditions, socioeconomic situations, and regional weather and climate. The proposed exposure-response model can then be assessed and compared to existing approaches through a simulation study and sensitivity analysis. Root mean squared error and interval coverage can be calculated from the simulated models in line with the methods in \cite{Hoskovec2021ModelStudy} and sensitivity analysis can be performed for the models' sensitivity to the inclusion of certain covariates, modelling assumptions (such as the buffer distances and any stationarity assumptions), and to identify particularly sensitive demographic groups to the effect's of air pollution exposure. \cite{Krewski2005ReanalysisAnalysis} undertakes a similar analysis for long-term exposure to fine particulate matter and sulfate-based air pollution on mortality.

\subsubsection{Objective 4}
An extension of exposure-response models is to casual models with Causal Expsoure-Repsonse Functions (CERFs), where an analysis of the evidence of causality can be used to more accurately represent the real-world effects of air pollution exposure on mental health outcomes. Assessing the evidence of causality in epidemiology can be approached in numerous ways, a traditional criteria for evidence of causality comes from Sir Austin Bradford Hill in 1956 \citep{Hill1965TheCausation}: Strength, Consistency, Specificity, Temporality, Biological Gradient, Plausibility, Coherence, Experiment, and Analogy. More modern approaches have since been devised, with a similar approach involving Directed Acyclic Graphs (DAGs) set out in \cite{Shimonovich2021AssessingThinking} or Empirical Dynamical Models (EDMs), often used in environmental modelling, as in \cite{Wu2020EmpiricalChina}. I will assess each indicator of causality using a variety of techniques: using statistical measures, biological context, and existing evidence. For example, reviewing the current evidence, I hope to identify any plausible biological pathways that could show how air pollution can adversely effect mental health outcomes for some indication of causality, such as inflammatory pathways discussed in \cite{Bakolis2021MentalSurvey}. As for strength, temporality and biological gradient, these can be assessed using common statistical methods and by assessing the exposure-response function. The investigation of spatial causal effects is a particular novel area of causal research and extending the notions of assessing causality to spatiotemporal data may open up new or extended areas of casual inference \citep{Akbari2021SpatialInference}. Then using any evidence that is found for causation, appropriate CERFs can be defined in a similar way to the original ERFs, examples of these can be seen in \cite{Ren2021BayesianOutcomes} and \cite{Papadogeorgou2020AMatter}, where development and assessment of the CERFs is analogous to the ERF case. In a similar vein to the exposure-response model, the proposed new CERFs and modelling framework will incorporate many of the cutting-edge approaches for multi-exposures, space-time effects and interactions, and the surrounding contextual covariates. 

Another important consideration in these models is the causal evidence behind the effect modifiers and potential confounders. It will be interesting to explore the causation arguments for the proximity to greenspaces and the previous evidence for the effects of underlying health, genetics, and socioeconomic status on mental health. This analysis and inference should give insight into the realistic effects of these covariates and how they tie into the relationship between air pollution and mental health. Finally, by considering the causal evidence and effect, the model can more accurately represent the real world situation in context to be used in prediction and policy assessment. From this, direct causal effects of air pollution can be measured, and as part of this project,  trends in vunerable groups and areas can be identified. Some papers that have investigated the proposed connection between causal inference and public health policy include \cite{Glass2013CausalHealth} and \cite{Pan2016HealthIntervention}.

\subsection{Visualisation}
For the visualisation of the resulting models many R packages exist, usually a combination of techniques can be used to view the data in various ways that may describe the data differently: \texttt{RQGIS} is useful for bridging R to GIS software and maps; \texttt{ggplot2} is a flexible package for visualising most types of data and can be used to create complex plots, maps and time series. Similarly, \texttt{shiny} can be used to develop interactive web applications for visualising and exploring dynamic maps.

\subsection{Training}

To complete this project using this methodology, I need to develop skills in a range of different disciplines. Firstly, I undertook the NASA Applied Remote Sensing Training (ARSET) Programme to develop knowledge and skills for the handling of satellite data, both the streams for land applications and air quality sensing. Complementary to this, I completed a course on the basics of JavaScript and on JavaScript for Google Earth Engine. As well as an online course in Machine Learning for Data Science for the underlying methods behind supervised and unsupervised classification for land use. This training will form the basis for the GIS models of air pollution and land use.

For the handling of the large datasets, I will be following the MSc course in Big Data from the Imperial College London Department of Mathematics. For the latter parts of the project, an increased knowledge pool of R packages for modelling, simulation, and visualisation will be useful and many online resources are available for this. Similarly, knowledge of more advanced and modern techniques in Bayesian spatiotemporal modelling will be helpful, especially in areas of evaluation and prediction, this will come from recent literature and comparisons of methods.