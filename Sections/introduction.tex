\chapter{Introduction}

\section{Motivation} \label{Section:Motivation}
    %- General motivation statement\\
    %- Air Pollution global and UK targets and aims\\
    %- Air pollution and physical health \\
    %- Mental health epi\\
    %- Air POllution and mental Health\\
    
It has been long established that exposure to air pollution and its relationship to climate change are having a significant adverse effect on physical health \citep{2021AirHealth}. This area of research, looking at identifying, quantifying and predicting these relationships, has a rich history of approaches and applications. Primarily the research has looked at the risk of exposure to three of the so-called criteria air pollutants, particulate matter (PM), Ozone (O\textsubscript{3}), and Nitrogen Dioxide (NO\textsubscript{2}), on cardiovascular health \citep[e.g.][]{Rajagopalan2018AirReview}, respiratory health \citep[e.g.][]{Saleh2020AirReview} and general mortality \citep[e.g.][]{Anderson2009AirHistory}. These do represent some of the recent pressing global health issues, but in this rapidly changing world the air pollution situation and current health priorities have developed. There is movement towards utilising modern measurement techniques of a wide variety of air pollutants and employing the use of new, large datasets and data handling techniques in health data analytics. While it is unequivocal that air pollution represents a threat for physical human health, it remains challenging to quantify its impact on mental health, and the concomitants synergies with changing in climate and socio-environmental factors.

The World Health Organization (WHO) has recently proposed the new 2021 Global Air Quality Guidelines (AQGs) \citep{WorldHealthOrganization2021WHOMonoxide} following increased quality and quantity of evidence for the effects of air pollution on different aspects of health since the 2005 AQGs. Subsequent to a comprehensive literature review, steps have been taken to determine a long-term concentration of each identified pollutant for which all non-accidental mortality and specific relevant health outcomes are "minimally relevant", i.e. the maximal pollutant concentration for which the relative risk is statistically insignificant. This concentration is take to be the AQG level and interim targets are based on a linear increase of relative risk for all non-accidental mortality.

These guidelines aim to be a global reference for a large audience of people, including policy makers, national and local authorities and organisations, and academics and researchers in the field \citep{WorldHealthOrganization2021WHOMonoxide}. 


\begin{table}[h!]
\centering
\begin{tabular}{c c c c c c c} 
 & & \multicolumn{4}{c}{Interim target} \\ [0.5ex] 
 \cline{3-6}
 Pollutant & Units & 1 & 2 & 3 & 4 & AQG level \\ [0.5ex] 
 \hline
 PM\textsubscript{2.5} \tablefootnote{fine particulate matter with a diameter of 2.5 micrometers ($\mu$m) or less} & $\mu$g/m\textsuperscript{3} & 35 & 25 & 15 & 10 & 5 \\ 
 PM\textsubscript{10} \tablefootnote{coarse particulate matter with a diameter of 10 micrometers ($\mu$m) or less} & $\mu$g/m\textsuperscript{3} & 70 & 50 & 30 & 20 & 15 \\ 
 NO\textsubscript{2} & $\mu$g/m\textsuperscript{3} & 40 & 30 & 20 & - & 10 \\ [0.5ex] 
 \hline
\end{tabular}
\caption{2021 WHO Recommended Annual AQC level and interim targets.}
\label{table:1}
\end{table}

On a national scale, the UK Department for Environment, Food \& Rural Affairs (Defra) has recently released the "world-leading" Environment Act 2021 \citep{DepartmentforEnvironmentFoodRuralAffairs2021World-leadingLaw}, with enforced updated environmental targets and legislation to protect and improve the UK's natural landscape. Following a public consultation this year, it is expected that the required target for air quality, specifically the annual mean concentration for ambient PM$_{2.5}$, will be brought closer in line with the WHO AQGs \citep{DefraPressOffice2022EnvironmentTargets} based on scientific evidence resulting from Defra's call for evidence \citep{DepartmentforEnvironmentFoodRuralAffairs2020CallTargets}. The independent Air Quality Expert Group (AQEG) have responded with a summary of recommendations for the future of modelling PM$_{2.5}$ \citep{AirQualityExpertGroup2021ModellingProcess}, including:
\begin{itemize}
    \item Developing the ability of models to include the effect of meteorological conditions on the dispersion of particulates;
    \item Improving the confidence of models to estimate and predict the future reduction of population exposure;
    \item Remove the reliance of models on ambient monitoring data through data fusion and model calibration.
\end{itemize}

Similarly, the Committee on the Medical Effects of Air Pollutants (COMEAP) have been asked to proved advice on the medical evidence for the PM$_{2.5}$ targets \citep{CommitteeontheMedicalEffectsofAirPollutants2021AdviceTargets}, including:
\begin{itemize}
    \item An assessment on the cost-benefit quanitifcation of health effects;
    \item Recommendations for the coefficients for mortality and hospital admissions associated to exposure;
    \item A comparison between the long-term and short-term exposure effects;
    \item The development of a health evidence based concentration limit value for long-term exposure to PM$_{2.5}$;
    \item Identifying groups within the population are more exposed and more susceptible to higher air pollution. 
\end{itemize}

Looking more closely at the current global health surveillance situation, many national and international organisations are calling for more data collection, research and resource allocation towards monitoring, preventing and treating the mental health difficulties of the world's population. Despite decades of research on the effects of air pollution exposure on human physical health, evidence and support for the effects on mental health are still limited. Early preliminary findings presented a potential mechanism of effect through the blood-brain barrier, ultrafine particles were found to cross the barrier and potentially enter the central nervous system \citep{Oberdorster2002UltrafineBeyond}, giving rise to potential adverse neurological and mental effects.

Looking forwards in the area of mental health research, the World Health Organisation (WHO) recently released an updated 'Comprehensive Mental Health Action Plan 2013-2030' \citep{WorldHealthOrganisation2021Comprehensive2013-2020}, identifying four main objectives including \emph{``to strengthen information systems, evidence and research for mental health''} particularly in the context of vulnerable populations and for identifying the major risk factors. Complementary, a core health focus of the Medical Research Council (MRC) \citep{MedicalResearchCouncil2019Delivery2019} is mental health, specifically \emph{``Improving understanding of the biological, social, and environmental risk factors of mental ill health with an emphasis on childhood and adolescence to enable early interventions.''} At a governmental-level, the newly formed UK Office for Health Improvement and Disparities' first initiative aims to improve the nation's mental health through national campaigns and by increasing awareness \citep{OHID2021NewHealth}.

For the general area of exposure epidemiology and parallel to the most recent UN Change Conference, COP26, many organisations are turning more of their attention to this pollution crisis, climate change and the burden it is having on the world's health. The 2021-2022 UK Health Security Agency's strategic remit \citep{Bethell2021UKPriorities} reflects this through the third of its four priorities for the next year: \emph{``Take action internationally to strengthen global health security, including ensuring the government has high-quality technical input in delivering its wider international health protection priorities, by supporting delivery of our G7 and COP26 priorities on improving global systems for disease surveillance and pandemic preparedness [...]''}.

The evidence-base for the relationship between increased air pollution concentrations, climate change and mental health is now rapidly increasing. Reflecting on a recent briefing paper by the Imperial College London Grantham Institute \citep{Lawrance2021ThePractice}, the detrimental effect of the climate crisis on mental health is a documented phenomenon but more research is required to understand \emph{``How and to what extent the climate crisis impacts populations’ mental health, who is affected, and to identify corresponding stressors, protective factors, support needs, and at risk groups''} \citep{Lawrance2021ThePractice}. This focus on finding spatiotemporal trends, socioeconomic patterns and other related variables brings a new layer to this area of research. To approach this novelty, an interdisciplinary approach will help to build on the foundations established in environmental exposure modelling and spatiotemporal epidemiological modelling, while also considering the social and individual context within the statistical models.

\section{Aims and Objectives}

The main aim of this PhD project is to quantify the long-term effect of air pollution on mental health outcomes in the UK, using residential proximity to greenspaces as an effect modifier.

The specific research objectives are:

\textbf{Objective 1:} Develop a novel spatiotemporal model for the fusion of the Pollution Climate Mapping (PCM) model with ground monitoring data and satellite-based air pollution measurements for the UK for the period 2005-2020. The outcome will be a monthly exposure model for fine particulate matter (PM\textsubscript{2.5}) at a spatial resolution of 1km by 1km. 

\textbf{Objective 2:} Develop an exposure-response model for evaluating the effect of air pollution on mental health outcomes, with land cover classification of greenspaces as a potential effect modifier and using other climate measures as potential cofounding factors. I will disentangle also the role of individual-level characteristics, such as demographic factors, lifestyles, personal and family medical history and physical measures, as potential confounderers.  

\textbf{Objective 3:} Causal inference analysis of the relationship between air pollution and mental health outcomes, with proximity to greenspaces as an effect modifier and taking into account spatial and temporal dependencies for the development of a novel causal exposure-response model.

\section{Structure of Report}

%Each project objective is considered separately in Chapters 2 to 4, each presented in the form of individual scientific paper proposals and progress reports, for the later aim of publication.

The current work on Objective 1 is outlined in Chapter 2, broadly presented as a scientific paper, with an emphasis on data and modelling choices, current progress and next steps. The progress and approaches towards Objectives 2 and 3 are considered in Chapter 3. Chapter 4 summarises the current work and results, as well as discusses the application and significance of the project, and project timeline. Finally, Chapter 5 is on the wider areas of the PhD and personal development.