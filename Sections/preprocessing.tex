\chapter{Data Preprocessing} \label{App:Preprocessing}

Ground monitoring data is obtained as point-referenced data at the UK monitoring stations, with specified site name, the downloads were reformatted to be read into R and can be matched to Defra's information on monitoring stations by site name, for exact coordinates, environment type, and region. Missing data??? \% of measurements???

The 11,682 granules of MCD19A2 level 2 gridded satellite-derived aerosol optical depth is bulk downloaded as .hdf files, which can be extracted into R using the gdalUtils package and converted to raster layers for: Aerosol Optical Depth at 047 micron; Aerosol Optical Depth at 055 micron; and AOD Uncertainty at 047 micron. Which can be aggregated to mean rasters for each variable by calendar month. The raw data is on a 1km x 1km gridded sinusoidal projection, and can be transformed to a 1km x 1km gridded Universal Transverse Mercator (UTM) projection, with EPSG:27700, also known as the British National Grid (BNG) or OSGB 1936. See GitHub repository \ref{ 

Across the entire time period, three days were missing MCD19A2 files for the required swaths: 03/10/2005, 29/03/2006, 31/12/2020. Available documentation suggests that these days are due to technical errors or time determination, so can be considered missing at random and removed from the monthly aggregation. 

The Pollution Climate Mapping (PCM) model .csv files can be simply read into R and converted to a raster from the x-y coordinates and modelled value. The coordinates are given on the 1km x 1km BNG.

For the meterological variables, the .nc files can be imported into R using ncvar\_get, converted to rasters and assigned the 1km x 1km BNG.

BLH??? and adjusting AOD? \cite{He2021TheAOD}
ERA5 Boundary Layer Height
relation to humidity \cite{He2019AGuangzhou} , pblh \cite{He2019AGuangzhou} 