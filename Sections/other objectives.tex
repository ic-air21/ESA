\chapter{Health Model and Causality Analysis}
\section{Exposure-Response Model}
\subsection{Background and Aims}
Chapter \ref{Section:Motivation} motivates this study into the effect of personal exposure to PM$_{2.5}$ on mental health outcomes and disorders, and summarises key stakeholders and questions involved in this area of research, but to begin to disentangle this relationship one must look to the current methods in both exposure-health modelling and mental health epidemiology.

Exposure assessment has a rich history, especially for use in epidemiological and health research, with numerous different study designs developed to measure or estimate some aspect of exposure. According to Daniel Vallero's \emph{`Fundamentals of Air Pollution`}, exposure assessment refers to "the process of characterizing, estimating, measuring, and modeling the magnitude, frequency, and duration of contact with an agent as well as the number and characteristics of the population exposed" \citep{Vallero2014AirExposures}. The agent of study is usually something with a potentially harmful effect, such as chemicals, biohazards, radiation, or pollutants. As mentioned previously, many different air pollutants could be considered, with PM$_{2.5}$ selected for its global interest, multiple sources, and previous evidential health effects. 

According to \cite{Han2017HumanPollution} there are two main categories of air pollution exposure assessment: direct and indirect methods. Direct methods include personal monitoring, i.e. carrying personal pollution monitors, and biological monitoring, i.e. monitoring personal biological samples, such as blood, to measure personal exposure. Whereas, indirect methods include environmental monitoring and questionnaire-based estimation. Direct methods usual provide a more accurate and precise measurement of exposure, but are expensive, time-consuming, and are more likely to introduce sample bias, whereas indirect methods can consider larger sample sizes and spatial domains at a time, but are more of an estimation of exposure. The indirect environmental monitoring methods is selected for this project for it's flexibility to capture UK-wide exposure to air pollution, that can be matched to various cohort or population-sampling studies. \cite{Jerrett2004AModels} discusses further some of the fundamental approaches to the environmental modelling method for air pollution and displays examples of applications and links to health effects. 

%Proximity models, such as those seen in \cite{Hoek2002AssociationStudy} and \cite{Janssen2001AssessmentMotorways}, consider measures of road type or traffic density in a buffer area around residential, school, or work addresses, and commonly link these measurements to exacerbation of asthma symptoms. 

The approach in Objective 1 to air pollution estimation is a modelled monthly gridded surface covering the UK for 2005-2020, with the aim that personal exposures can be estimated monthly by the grid cell surrounding their residential address. The 1 x 1km grid squares are at a finer spatial resolution than many current studies, to evaluate finer changes in pollutant sources, meteorology, and spatial structure. Whereas the large time period and monthly time intervals, allow for a wider application of the data, up to 16 years of long-term exposure can be estimated, as well as more short-term episodes and changes in exposure; in this case, the aim is to use a long-term exposure estimate approximately 10 years prior to an individuals initial data collection date or mental health diagnoses. Grid based approaches are also notably not effected by regional aggregation and sometimes arbitrary area boundaires, although at a 1 x 1km scale, these grid cells can be aggregated to national or subnational areas \citep{Anenberg2016SurveyTools}.

current air poll and mental health, bayesian

As previously discussed, the global work on the identification and quantification of the relationship between air pollution and mental health is rapidly increasing, with many governmental, non-profit, and academic organisations devoting more and more work, time, and effort into this increasingly important field. The prominent briefing paper \cite{Lawrance.Emma2021ThePractice} highlights this disparity between global awareness and lack of publications on the topic. 

This project may benefit from a more thorough literature review, at a later date, but here I consider 20 recent studies and their methods for investigating this negative effect of air pollution exposure on mental health:

\begin{landscape}
\begin{longtable}{c c p{4cm} p{6cm}}
\textbf{Source} & \textbf{Air Pollutants} & \textbf{Mental Health Outcome} & \textbf{Modelling Methods} \\



\end{longtable}
\end{landscape}


The overall aim of Objective 2 is to create an exposure-response model to quantify the relationship between exposure to fine particular matter (PM$_2.5$) and mental health outcomes within the UK. More broadly speaking, there are a number of underlying research questions to be considered:
\begin{enumerate}
    \item How to estimate a personal exposure level of individual participants in a cohort study?
    \item How to develop a Bayesian exposure-response model that considers a wide range of individual characteristics and incorporates uncertainty in exposure estimation?
    \item What effect does proximity to greenspaces have on the exposure-response relationship and how should it be included in the model?
    \item Which mental health diagnoses and outcomes are linked to increased air pollution exposure? Are there any significant links to specific symptoms or presentations?
    \item Is the relationship between air pollution and mental health robust after accounting for individual-level characteristics, such as demographic factors, lifestyles, personal and family medical history, and physical measures?
\end{enumerate}

%Another big recent development is the existence and availability of large biomedical datasets \citep{Ristevski2018BigHealthcare}. This has been seen across disciplines and has come with the rise of big data science techniques, such as digital data collection, cloud storage and big data analytics \citep{Hariri2019UncertaintyChallenges}, as well as a shift in perspective in the data world to more open access data and global data analysis \citep{Zhu2020Open-accessAcademia}. One main source of rich biomedical data in the UK is the UK Biobank, containing information on the genetics and health of half a million volunteer participants \citep{Collins2020UKBiobank}. Another, interesting dataset is the Millennium Cohort Study \citep{Joshi2016ThePolicy}, following the lives, family and health of around 19,000 young people across England, Scotland, and Wales. Most importantly for air pollution and mental health epidemiological modelling, both of these datasets include variables such as the participants' location, socioeconomic status, and mental health status, both clinical and subclinical measures

%Before incorporating the health data into the model, it is important to assess it separately. First, an assessment of the participant sample will help to give a contextual basis of our analyses within the UK population. I am interested in the makeup of the participants age, sex, location, socioeconomic status, occupation, ethnicity, lifestyle, health status etc., particularly for vulnerable or underrepresented groups. I am also interested in identifying any clear trends in the mental health outcome data from the offset, maybe spatially, temporally or demographically, that can then potentially be explained via the air pollution or meterological variables. As for the definition of mental health outcomes/disorders, corresponding to the available UK Biobank variables, I will use the self-reported mental health indicators for depression, anxiety, and bipolar, along with the physician-derived depression, anxiety, and bipolar statuses. These sub-clinical measures will be interesting to assess the underlying mental health of the population, compared to acute mental health events, such as given by the diagnosed disorders in the Biobank data using the International Statistical Classification of Diseases and Related Health Problems (ICD) codes \citep{WorldHealthOrganisationICD-10Version:2019} for inpatient hospital visits.

%Then for the exposure-response modelling, I will review and compare exposure-response estimation methods within a hierarchical Bayesian framework to then investigate a novel approach utilising the cutting-edge methods in multivariate spatiotemporal modelling with extensive contextual variables. I will start off by reviewing single-exposure effects, looking at each of the four pollutants separately and assessing the one-dimensional approaches first. A common basic approach is Generalised Linear Models (GLMs) with a linear Exposure-Response Function (ERF), such as the one seen in \cite{Shi2016Low-ConcentrationStudy}. The GLM approach can then be extended to non-linear Exposure-Response Functions (ERFs), for example, by using splines: \cite{Smith2000ThresholdArizona} gives an example of the use of B-splines and \cite{Daniels2000EstimatingCities} for cubic splines, both using a cross-validation method for the estimation of the splines. Generalised Additive Models (GAMs) are then a natural progression to more complicated relationship modelling, with similar linear and non-linear approaches available, again this can be seen in \cite{Smith2000ThresholdArizona}. Furthermore, I will investigate the lag effects of the exposure on the response, in a similar way to \cite{Shi2016Low-ConcentrationStudy}, with considerations on how we chose a lag through Goodness-of-fit tests as in \cite{Richardson2011LaggingAnalyses}. Finally, the extension to simultaneous exposure health effects should be considered, there are again a number of different approaches to this, but within the Bayesian framework I will mainly be interested in ERFs in regression models. An extension to multi-pollutant spline functions can be seen in \cite{Chen2013InfluenceChina} and a similar kernel function can be seen in \cite{Bobb2015BayesianMixtures}. The potential effect modifying effect of proximity to greenspaces can be easily incorporated into these regression models, where the effect of its inclusion can be assessed in a similar way to as in \cite{Ghosh2010PaternalMothers} and \cite{Mariet2021AssociationExposure}. Similarly, the previous considerations for the confounding and effect modifying properties of meteorological measurements (including temperature, humidity and wind) can be investigated through the model. For the purposes of this project, the air pollution and meteorological measurements are being considered as a proxy to direct measures of climate change: CO\textsubscript{2} and CO emissions are closely linked to NO\textsubscript{2} and SO\textsubscript{2} emissions \citep{Chen2007OutdoorEffects}; long-term temperature (or climatic temperature) increase is a common indication of global warming \citep[e.g.][]{Du2019ChangesHiatus}; and the levels of atmospheric Ozone are intricately linked to the greenhouse effect \citep{Meleux2007IncreaseChange}.

%The novelty of my approach will come from the combination of the existing approaches from all directions and the extension to the inclusion of underlying health conditions, socioeconomic situations, and regional weather and climate. The proposed exposure-response model can then be assessed and compared to existing approaches through a simulation study and sensitivity analysis. Root mean squared error and interval coverage can be calculated from the simulated models in line with the methods in \cite{Hoskovec2021ModelStudy} and sensitivity analysis can be performed for the models' sensitivity to the inclusion of certain covariates, modelling assumptions (such as the buffer distances and any stationarity assumptions), and to identify particularly sensitive demographic groups to the effect's of air pollution exposure. \cite{Krewski2005ReanalysisAnalysis} undertakes a similar analysis for long-term exposure to fine particulate matter and sulfate-based air pollution on mortality.

\subsection{Proposed Research Methods} \label{Health Methods}

specific questions???
model types?
mixture model?
scores? propensity scores?
longitudinal health?
lag?

...? Health?\\

\subsubsection{Biobank Data}
The first proposed source of individual-health data is the UK Biobank database, a cohort study started in 2006 across the UK. It contains biological, health, and lifestyle data on around 500,000 people aged 40-69 years in England, Scotland, and Wales, and is collected from multiple assessments and follow-ups. Variables of interest include:
\begin{itemize}
    \item Personal data: sex, date of birth, country of birth, weight, height, index of multiple deprivation, housing score, health score, education score, cause and date of death
    \item Home location data: home location, home area population density, water/natural environment/greenspace percentages, nearby traffic intensity, noise pollution, physical environment score, living environment score.
    \item Employment and job data: employment score, income score, type of work, workplace exposures, job satisfaction
    \item Lifestyle: diet, alcohol consumption, tobacco and drug use, prescriptions, intensity and frequency of exercise, sleep health, time spent outdoors
    \item Physical health: diagnosed heart or lung problems, diagnosed diabetes, diagnosed cancer, diagnosed other serious medical condition/disability, other reported medical symptoms or disorders, operations
    \item Mental health: symptoms of anxiety, symptoms of stress, symptoms of depression, symptoms of mania, symptoms of schizophrenia/psychosis, sought medical help, diagnosed mental health disorders, neuroticism score, happiness, self harm and suicide attempts, derived mental health status
    \item Cognitive function: diagnosed developmental disorders and dementia, fluid intelligence score, matrix pattern completion, numeric memory, paired associate learning, pairs matching, prospective memory, reaction time, symbol digit substitution, tower rearranging, trail making.
\end{itemize}

The UK Biobank has provided a rich source of data for many related studies (see Appendix \ref{App:Biobank studies}).

\subsubsection{SCAMP Data}
The Study of Cognition, Adolescents and Mobile Phones (SCAMP) is a cohort study of 7000 adolescents from 39 schools in Greater London, including data on demographics, lifestyle, environment, and health. The study started in 2014, with follow-up data collection every 2 years, and was primarily collected to investigate \emph{`the impact of mobile phones and social media on young people’s physical and mental health, and brain function`}, but is also being used in other related substudies (see Appendix \ref{App: SCAMP studies}).

The data collected from the school assessments and online questionnaire include:
\begin{itemize}
    \item Sociodemographics: age, sex, height, weight, parental education, home address.
    \item Home environment: travel to school, noise exposure, use of green and blue spaces, home conditions.
    \item Technology use: mobile phones, desktop computers, laptops, tablets, gaming consoles, internet, social media.
    \item Health and Wellbeing: Health-related quality of life, sleep health, medical symptoms and prescriptions, learning disabilities, symptoms of depression and anxiety, behaviour, physical activity, diet.
    \item Cognitive assessment: fluid intelligence, speech processing, cognitive flexibility, sustain attention, inhibition, working memory, visual attention, mental rotation.
\end{itemize}


\subsubsection{Exposure-Response Model}
longitudinal mixed model cohort study? like MESA Air study

\subsection{Current Progress}
Getting the Biobank dataset, any data cleaning, other datasets

\subsection{Next Steps}

\subsection{Expected Results and Significance}

\section{Causality Analysis}
\subsection{Abstract}

\subsection{Background and Aims}
\cite[p.550]{Gelfand2019HandbookStatistics}

%An extension of exposure-response models is to casual models with Causal Expsoure-Repsonse Functions (CERFs), where an analysis of the evidence of causality can be used to more accurately represent the real-world effects of air pollution exposure on mental health outcomes. Assessing the evidence of causality in epidemiology can be approached in numerous ways, a traditional criteria for evidence of causality comes from Sir Austin Bradford Hill in 1956 \citep{Hill1965TheCausation}: Strength, Consistency, Specificity, Temporality, Biological Gradient, Plausibility, Coherence, Experiment, and Analogy. More modern approaches have since been devised, with a similar approach involving Directed Acyclic Graphs (DAGs) set out in \cite{Shimonovich2021AssessingThinking} or Empirical Dynamical Models (EDMs), often used in environmental modelling, as in \cite{Wu2020EmpiricalChina}. I will assess each indicator of causality using a variety of techniques: using statistical measures, biological context, and existing evidence. For example, reviewing the current evidence, I hope to identify any plausible biological pathways that could show how air pollution can adversely effect mental health outcomes for some indication of causality, such as inflammatory pathways discussed in \cite{Bakolis2021MentalSurvey}. As for strength, temporality and biological gradient, these can be assessed using common statistical methods and by assessing the exposure-response function. The investigation of spatial causal effects is a particular novel area of causal research and extending the notions of assessing causality to spatiotemporal data may open up new or extended areas of casual inference \citep{Akbari2021SpatialInference}. Then using any evidence that is found for causation, appropriate CERFs can be defined in a similar way to the original ERFs, examples of these can be seen in \cite{Ren2021BayesianOutcomes} and \cite{Papadogeorgou2020AMatter}, where development and assessment of the CERFs is analogous to the ERF case. In a similar vein to the exposure-response model, the proposed new CERFs and modelling framework will incorporate many of the cutting-edge approaches for multi-exposures, space-time effects and interactions, and the surrounding contextual covariates. 

%Another important consideration in these models is the causal evidence behind the effect modifiers and potential confounders. It will be interesting to explore the causation arguments for the proximity to greenspaces and the previous evidence for the effects of underlying health, genetics, and socioeconomic status on mental health. This analysis and inference should give insight into the realistic effects of these covariates and how they tie into the relationship between air pollution and mental health. Finally, by considering the causal evidence and effect, the model can more accurately represent the real world situation in context to be used in prediction and policy assessment. From this, direct causal effects of air pollution can be measured, and as part of this project,  trends in vunerable groups and areas can be identified. Some papers that have investigated the proposed connection between causal inference and public health policy include \cite{Glass2013CausalHealth} and \cite{Pan2016HealthIntervention}.


\subsection{Proposed Research Methods}

\subsection{Expected Results and Significance}